\PassOptionsToPackage{unicode=true}{hyperref} % options for packages loaded elsewhere
\PassOptionsToPackage{hyphens}{url}
%
\documentclass[letterpaper]{article}
\usepackage{lmodern}
\usepackage{xcolor}
\usepackage{fontspec}
\usepackage{minted}
\usepackage{amssymb,amsmath}
\usepackage{ifxetex,ifluatex}
\usepackage[margin=1in]{geometry}
\usepackage{fixltx2e} % provides \textsubscript
\ifnum 0\ifxetex 1\fi\ifluatex 1\fi=0 % if pdftex
  \usepackage[T1]{fontenc}
  \usepackage[utf8]{inputenc}
  \usepackage{textcomp} % provides euro and other symbols
\else % if luatex or xelatex
  \usepackage{unicode-math}
  \defaultfontfeatures{Ligatures=TeX,Scale=MatchLowercase}
\fi
% use upquote if available, for straight quotes in verbatim environments
\IfFileExists{upquote.sty}{\usepackage{upquote}}{}
% use microtype if available
\IfFileExists{microtype.sty}{%
\usepackage[]{microtype}
\UseMicrotypeSet[protrusion]{basicmath} % disable protrusion for tt fonts
}{}
\IfFileExists{parskip.sty}{%
\usepackage{parskip}
}{% else
\setlength{\parindent}{0pt}
\setlength{\parskip}{6pt plus 2pt minus 1pt}
}
\usepackage{hyperref}
\hypersetup{
            pdfborder={0 0 0},
            breaklinks=true}
\urlstyle{same}  % don't use monospace font for urls
\usepackage{longtable,booktabs}
% Fix footnotes in tables (requires footnote package)
\IfFileExists{footnote.sty}{\usepackage{footnote}\makesavenoteenv{longtable}}{}
\usepackage{graphicx,grffile}
\makeatletter
\def\maxwidth{\ifdim\Gin@nat@width>\linewidth\linewidth\else\Gin@nat@width\fi}
\def\maxheight{\ifdim\Gin@nat@height>\textheight\textheight\else\Gin@nat@height\fi}
\makeatother
% Scale images if necessary, so that they will not overflow the page
% margins by default, and it is still possible to overwrite the defaults
% using explicit options in \includegraphics[width, height, ...]{}
\setkeys{Gin}{width=\maxwidth,height=\maxheight,keepaspectratio}
\setlength{\emergencystretch}{3em}  % prevent overfull lines
\providecommand{\tightlist}{%
  \setlength{\itemsep}{0pt}\setlength{\parskip}{0pt}}
\setcounter{secnumdepth}{0}
% Redefines (sub)paragraphs to behave more like sections
\ifx\paragraph\undefined\else
\let\oldparagraph\paragraph
\renewcommand{\paragraph}[1]{\oldparagraph{#1}\mbox{}}
\fi
\ifx\subparagraph\undefined\else
\let\oldsubparagraph\subparagraph
\renewcommand{\subparagraph}[1]{\oldsubparagraph{#1}\mbox{}}
\fi

% set default figure placement to htbp
\makeatletter
\def\fps@figure{htbp}
\makeatother

\bibliographystyle{abbrv}

\date{}

\begin{document}

\flushbottom
\thispagestyle{empty}

\section*{ROV Reference Policy and Custom Policies}

\subsection*{Route Origin Validation}

Route Origin Validation (ROV) is a mechanism designed to prevent unplanned
advertisement of routes. It uses Resource Public Key Infrastructure (RPKI) to
make sure that an advertisement originates from a valid AS. Specifically, ROV
queries a database received from an RPKI cache server that contains
prefix-to-AS mappings. An announcement is valid if its origin AS and prefix are
found in the database \cite{Juniper}. ROV Extrapolator simulates Route Origin
Validation.

\subsection*{Creating a Custom Policy Extrapolator}

To create a new extrapolator that implements a custom policy, it is recommended
to inherit from every class that the extrapolator uses. Specifically, a child
should be created from each of these classes: 

\begin{itemize}
\item Announcement
\item BaseAS
\item BlockedExtrapolator
\item BaseGraph
\item SQLQuerier
\end{itemize} 

\subsubsection*{Announcement}

If the desired policy uses additional properties or methods for announcements,
they should be added to the children of the Announcement class. Although it
does not use any additional variables, ROVAnnouncement class provides a good
example of inheritance from the Announcement class.

\subsubsection*{AS}

Most likely, the AS class for a new policy will override some methods of the
BaseAS class. For example, ROVAS overrides the \texttt{process\_announcement}
method to reject announcements per ROV as well as defines four new functions to
handle ROV.

\subsubsection*{Extrapolator}

The new extrapolator will inherit from BlockedExtrapolator and use other
classes mentioned in this section. This class is responsible for extrapolation,
seeding announcements, and several other things. In the case of
ROVExtrapolator, \texttt{extrapolate\_blocks} is overridden to keep track of
announcements' ROA validity. Furthermore, the ROV version overrides
\texttt{give\_ann\_to\_as\_path} so that announcements are seeded with ROV
taken into account. Specifically, this function sets
\texttt{received\_from\_asn} of the announcement's origin AS to 64513 when it
is not valid and 64514 when it is valid. These ASNs are private, so there will
be no interference during extrapolation. Later in lib\_bgp\_data, those numbers
enable traceback in SQL. 

\subsubsection*{Graph}

ROVASGraph class can be used as a reference for creating a new Graph class. By
default, the \texttt{process} function in BaseGraph removes stubs. That is done
because simulation scenarios that do not seed announcements at edge ASes may
want to keep the stubs removed since all stubs will have the same routing
information bases as their providers. In most cases, the new Graph class should
override it because keeping stubs is preferred for achieving a greater level of
detail and for cases when a stub is an attacker. When data from additional
table(s) is needed to create the graph, \texttt{create\_graph\_from\_db} should
be overridden. Additionally, Graph implements the \texttt{createNew} function
that creates a new AS. If the new AS class has a constructor with additional
parameters, that function should be overridden as well. 

\subsubsection*{SQLQuerier}

SQLQuerier is the class made for interaction with the database. In cases when
the new extrapolator utilizes tables that differ from the default ones, new
functions should be defined in SQLQuerier's child. For example, ROVSQLQuerier
overrides \texttt{select\_prefix\_ann} and \texttt{select\_subnet\_ann}
functions because the ROV version needs to select an additional field
containing ROA validity. Furthermore, this class implements the
\texttt{select\_AS\_flags} function (used in
\texttt{ROVASGraph::create\_graph\_from\_db}) to get ROV adoption for each AS.

\bibliography{security_policies}

\end{document}
